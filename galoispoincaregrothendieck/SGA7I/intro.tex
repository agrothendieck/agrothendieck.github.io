%Begin






%%%%%%%%%%%%%%%%%%%%%%%%%%%%%%%%%%%%%%%%%%%%%%%%%%%%%%%%%%%%%%%
%INTRODUCTION







Soient $S$ un schéma et $f: X \to S$ un morphisme de schémas. Si $f$ est 
propre et lisse, et que le nombre premier $\ell$ est inversible sur $S$, les groupes de
cohomologie $\ell$-adique $\HH^i(X_{\overline{s}}, \mathbf{Z}_{\ell})$ des fibres géométriques
de $f$ forent un système local $\ell$-adique sur $S$. Pour $f$ seulement supposé propre, ces groupes sont
les fibres d'un faisceau $\ell$-adique $\RR^if_*\mathbf{Z}_{\ell}$ sur $S$. La \emph{théorie des
cycles évanescents} met en relation la ramification de ce faisceau sur $S$ et les
singularités de $f$.

Nous ne considérerons que le cas où $S$ est un trait hensélien ( = spectre d'un anneau
de valuation discrète hensélien). C'est en pratique le cas essentiel. Je renvoie à (I 2.1) pour une
description heuristique de la théorie. Soient $S = \Spec(V)$, $k(\overline{\eta})$ une clôture 
algébrique du corps des fractions $k(\eta)$ de $V$ et $k(\overline{s})$ la clôture algébrique
correspondante du corps résiduel $k(s)$ ($k(\overline{s})$ est le corps résiduel du normalisé 
$\overline{V}$ de $V$ dans $k(\overline{\eta})$). Dans le cas particulier où $X$ est propre et plat
sur $S$, de dimension relative $n$, et où $f$ ne présente qu'un point de non lissité $x \in X_S$,
on définit des $Gal(\overline{\eta}/\eta)$-modules $\phi^i$, de nature purement locale au
voisinage de $x$, nuls pour $i \notin [0, n]$ (I 4.2), et on construit une suite exacte longue de
$Gal(\overline{\eta}/\eta)$-modules
$$ ... \to \HH^i(X_{\overline{s}}, \mathbf{Z}_{\ell}) \xlongrightarrow{sp} \HH^i(X_{\overline{\eta}}, 
\mathbf{Z}_{\ell}) \to \phi^i \to \HH^{i+1}(X_s, \mathbf{Z}_{\ell}) \to ... $$
(le $Gal(\overline{s},s)$-module $\HH^i(X_{\overline{s}}, \mathbf{Z}_{\ell})$ est regardé comme un
$Gal(\overline{\eta}/\eta)$-module avec action triviale de l'inertie ; $sp$ est la flèche de
spécialisation).

On donne aussi des critères, valables pour l'instant seulement en caractéristique $0$, pour que les
$\phi^i$ pour $i$ petits soients nuls (par exemple, si $X$ est de plus localement d'intersection complète,
$\phi^i = 0$ pour $i \neq n$ (I. 4.5)).

Le \emph{théorème de monodromie} affirme que l'action du sous-groupe d'inertie $I$ de
$\Gal(\overline{{\eta}}/\eta)$ est quasi-unipotente : pour $T \in I$, il existe des entiers $N>0$,
$M>0$ tels que l'endomorphisme $(T^M - I)^N$ de $\HH^i(X_{\overline{\eta}}, \mathbf{Z}_{\ell})$
soit nul. On donne de ce théorème deux démonstrations. La première (I.1.2), de nature arithmétique,
s'applique dès que les groupes de cohomologie considérés ont des propriétés de finitude raisonnables.
Le point clef est que, lorsque $k(s)$ est de type fini sur son sous-corps premier, l'action de $I$ est
quasi-unipotente pour \emph{toute} représentation $\ell$-adique de $\Gal(\overline{{\eta}}/\eta)$.
La seconde démonstration, plus géométrique, requiert la pureté et la résolution des singularités : 
elle n'est pour l'instant valable qu'en caractéristique $0$ ou pour un $\HH^1$. Elle apporte de 
précieuses informations sur l'exposant de nilpotence $N (N \leq i+1$ pour un $\HH^i)$. Ainsi qu'on le
verra ultérieurement, elle se prête bien, sur $\mathbf{C}$, à la comparaison avec la théorie 
transcendante.

Ces résultats, appliqués au $\HH^1$ des variétés abéliennes, i.e. à leur module de Tate, permettent 
d'étudier la réduction $mod p$ de celles-ci. On donne ainsi deux démonstrations du théorème de réduction 
stable des variétés abéliennes selon lequel, après ramification, la fibre spéciale connexe du modèle de 
Néron est extension d'une variété abélienne par un tore. La première (I. 6) reprend la méthode de la 
démonstration arithmétique du théorème de monodromie. La seconde (IX 3.6) s'appuie sur une analyse 
beaucoup plus fine du modèle de Néron, et des ses propriétés de polarisation.

Les exposés I à V de Grothendieck n'ont pas été rédigés. Ils ont été résumés dans un exposé I.
Les résultats énoncés y sont démontrés de fa\c{c}on succinte, mais essentiellement complète.

L'exposé II applique la méthode des pinceaux de Lefschetz et (I 5.3) à l'étude du groupe fondamental. 
Pour $S$ une surface sur un corps algébriquement clos $k$, d'exposant caractéristique $p$, on montre
que le groupe profini $\pi^{(p)}_1(S, s)$, complété en dehors de $p$ du groupe fondamental, est de 
pro-$(p)$-présentation finie.

Comme expliqué plus haut, les exposés III à V n'existent pas.

L'exposé VI contient, avec quelques compléments, la théorie des déformations de Schlessinger. On y prend
soin de tenir compte des automorphismes infinitésimaux des objets qu'on classifie, et de ne pas passer 
trop brutalement au foncteur des classes d'isomorphie d'objets. On y donne aussi une nouvelle construction
des $\underline{\Ext}^i(L_{X/S}, -)$ ($L_{X/S}$ complexe cotangent relatif de $X/S$). Cet exposé sert dans
le reste du séminaire surtout via l'étude qui y est faite des déformations des singularités quadratiques
ordinaires.

Les exposés VII et VIII sont consacrés à la théorie des biextensions. Cette théorie joue un rôle essentiel
dans l'exposé IX, pour exprimer ce qu'il advient d'une polarisation quand on passe d'une variété abélienne
à un modèle de Néron de celle-ci. Cet exposé IX contient la démonstration du théorème de réduction stable
des variétés abéliennes, et diverses applications.

La suite de ce séminaire : SGA 7 II, par P. Deligne et N. Katz, paraîtra ultérieurement.

\vskip .3cm

\begin{flushright}
Bures sur Yvette, mai 1972, \\ P. DELIGNE
\end{flushright}














%End

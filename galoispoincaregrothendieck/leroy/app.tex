%Begin







%%%%%%%%%%%%%%%%%%%%%%%%%%%%%%%%%%%%%%%%%%%%%%%%%%%%%%%%%%%%%%% APPENDICE: CATÉGORIES FIBRÉES EN TOPOS

Soit $I$ une catégorie $\cU$-petite.

\vskip .3cm
{\bf A.1.} J'appelle catégorie fibrée en $\cU$-topos au dessus de $I$ toute catégorie fibrée $F \xlongrightarrow{\Pi} I$ qui vérifie les axiomes suivantes : 
\begin{enumerate}
    \item[1)] Pour tout $i \in \Ob(I)$, la fibre $F_i$ est un $\cU$-topos. 
    \item[2)] Pour toute flèche $u: i \to j$ de $I$, le foncteur changement de base $F_j \to F_i$ définit un morphisme de topos $F_i \to F_j$.
\end{enumerate}

\vskip .3cm
{\bf A.2.} Définissons maintenant la $2$-catégorie $\Fibtop(I)$ des catégories fibrées en $\cU$-topos au-dessus de $I$ : étant données deux catégories fibrées en topos $F$, $G$ au-dessus de $I$, nous prenons comme catégorie des morphismes de $F$ dan $G$
$$
\Cartop_I(F, G)
$$
la sous-catégorie pleine de 
$$
\Cart_I(G, F)^{\circ}
$$
(catégorie opposée de la catégorie des $I$-foncteurs cartésiens $G \to F$) définie comme voici~: un $I$-foncteur cartésien $\phi: G \to F$ définit un morphisme de catégories fibrées en topos $F \to G$ si pour tout $i \in \Ob(I)$ le foncteur $G_i \to F_i$ déduit de $\phi$ par restriction définit un morphisme de topos $F_i \to G_i$.

\vskip .3cm
{\bf A.3.} La $2$-catégorie $\Fibtop(I)$ est équivalente à la $2$-catégorie des $2$-foncteurs de $I$ dans la catégorie des $\cU$-topos: les catégories fibrées de la forme $I \times E$ ($E$ un $\cU$-topos) correspondant aux $2$-foncteurs constants ; d'où une définition des $2$-limites inductives et projectives de topos~:

\vskip .3cm
{\bf A.4.} Soit $F$ une catégorie fibrée en $\cU$-topos au-dessus de $I$. Nous appellerons $2$-limite inductive de $F$ le $2$-foncteur covariant
\begin{align*}
 \cU\text{-topos} \to \cU\text{-catégories}\\
 E \mapsto \Cartop_I(F, I \times E)
\end{align*}
et $2$-limite projective de $F$ le $2$-foncteur contravariant
$$
E \to \Cartop_I(I \times E, F)
$$
La $2$-limite inductive (resp. projective) de $F$ se représente donc, quand c'est possible, par un $\cU$-topos $L$ muni d'un morphisme de catégories fibrées en topos
$$
F \to I \times L
$$
$$
\text{(resp.}~I \times L \to F).
$$







%End

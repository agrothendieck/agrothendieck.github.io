%Begin






%%%%%%%%%%%%%%%%%%%%%%%%%%%%%%%%%%%%%%%%%%%%%%%%%%%%%%%%%%%%%%% TABLE ANALYTIQUE

Le titre suffit à délimiter le sujet ; j'ai mis les explications indispensables dans la table des matières, formant ainsi une table analytique.

\subsection*{1. Objets connexes dans un topos}

Bref exposé des notions nécessaires pour définir un topos localement connexe.

\subsection*{2. Objets localement constants et objets galoisiens}

{\bf 2.1.} Les objets localement constants d'un topos correspondent aux revêtements d'un espace topologique ou d'un schéma, regardés comme des faisceaux.

{\bf 2.2.} On démontre pour les objets localement constants d'un topos localement connexe les principales propriétés des revêtements d'un espace localement connexe.

{\bf 2.3.} Les objets galoisiens correspondent aux revêtements galoisiens. La ``théorie de Galois'' classe les objets localement constants trivialisés par un objet galoisien donné d'un topos connexe. (Dans le topos étale du spectre d'un corps $k$, les objets galoisiens sont les extensions galoisiennes de $k$ ; on retrouve ainsi la théorie de Galois classique).

{\bf 2.4.} Topos engendré par les objets localement constants d'un topos localement connexe donné $E$ : les résultats du chapitre suivant permettront de regarder ce topos, qui est formé des sommes directes d'objets localement constants de $E$, comme le groupoïde fondamental de $E$.

\subsection*{3. Topos localement galoisiens et groupoïde fondamental}

La notion de topos localement galoisien nous tiendra lien d'une fastidieuse théorie des ``pro-groupoïdes''; et elle permet de définir le groupoïde fondamental d'un topos par une propriété universelle.

\subsection*{4. Limites inductives de topos et théorème de Van Kampen}

On définit un système inductif de topos à l'aide d'une catégorie fibrée en topos au-dessus d'une catégorie d'indices. Les sections cartésiennes de cette catégorie fibrée sont les objets du topos limite inductive du système. Ainsi les objets d'une limite inductive de topos apparaissent comme des objets de la somme directe munis d'une certaine donnée de descente. Le théorème 4.5. sert à décrire le groupoïde fondamental d'une limite inductive de topos localement connexes connaissant leurs groupoïdes fondamentaux. L'énoncé et le démonstration de ce théorème font intervenir un topos auxiliaire, sorte de recollement intermédiaire entre la somme directe et la limite inductive, qui est décrit en (4.3.) et (4.4.). Je l'ai éliminé dans le corollaire de la proposition 4.6.2., qui décrit directement les objets localement constants de la limite inductive. L'avantage de la forme (4.5.) est de permettre des calculs explicites, qui sont développées dans les poins 4.6.3. à 4.6.7. 

\subsection*{A. Appendice}

Catégories fibrées en topos

\subsection*{5. Compléments}

{\bf 5.1.} Groupe fondamental d'un topos localement connexe en un point.

{\bf 5.2.} Groupoïde fondamental profini.







%End

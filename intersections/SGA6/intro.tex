%Begin






%%%%%%%%%%%%%%%%%%%%%%%%%%%%%%%%%%%%%%%%%%%%%%%%%%%%%%%%%%%%%%%
%INTRODUCTION





Le but du présent Séminaire est de développer une théorie \emph{globale} des intersections, 
et une formule de Riemann-Roch, pour des schémas quelconques. Le lecteur trouvera une description 
du programme du Séminaire dans l’exposé $0$. La possibilité en principe d’une démonstration d’une 
formule de Riemann-Roch pour les schémas \emph{réguliers} généraux, suivant les lignes du rapport 
bien connu de Borel-Serre en 1958, était claire dès ce moment, du moins pour un morphisme projectif. 
L’extension au cas général est moins évidente ; le programme dans lequel elle s’insère (et partiellement 
réalisé dans notre séminaire) remonte à 1960. Comme c’était également le cas pour la théorie de 
dualité des faisceaux cohérents (cf. R. Hartshorne, Residues and Duality, Lecture Notes in Mathematics 
n$^\circ 20$, Springer), un outil essentiel pour formuler une théorie satisfaisante est la théorie des 
\emph{catégories dérivées} de Verdier, dont la connaissance est indispensable pour la compréhension du 
Séminaire.

A part cette théorie, l’étude du Séminaire n’exige guère qu’une connaissance générale des fondements 
de la Géométrie Algébrique, tels qu’ils sont exposés dans EGA, chapitres I, II, III ; nous n’aurons en 
plus que des références occasionnelles à faire à EGA IV, pour certains faits concernant les morphismes 
plats ou lisses, qui pour l’essentiel figurent également dans les premiers exposés de SGA 1. Enfin, 
pour développer certains résultats avec toute la généralité souhaitable pour les applications, nous 
faisons usage parfois de la notion de \emph{site annelé} et de \emph{topos annelé}, pour laquelle nous 
renvoyons à SGA 4, exposés I à IV. Le lecteur qui ignorerait le langage des sites et topos pourra 
remplacer partout lesdits objets par des espaces topologiques ordinaires, les objets du topos étant 
alors remplacés par des ouverts de ces espaces; mais nous lui conseillons néanmoins, de préférence, 
de s’assimiler le langage des topos, qui fournit un principe d’unification extrêment commode.

La notion fondamentale pour la théorie présentée ici est celle de \emph{complexe de Modules parfait}, 
et ses diverses variantes, développées dans les exposés I, II, III. Il semble clair que ces notions, 
importantes également dans d’autres contextes en Géométrie Algébrique (notamment pour les formules de 
type Lefschetz-Verdier en cohomologie à coefficients discrets (SGA 5) ou cohérents), auront aussi leur 
rôle à jouer en dehors de la Géométrie Algébrique, notamment pour la formulation d’une variante analytique 
complexe du théorème de Riemann-Roch présenté ici, ou de variantes convenables du théorème de l’index 
d’Atiyah-Singer. Quelques indications dans ce sens seront données dans Exp. II. Malheureusement, il 
manque encore à l’heure actuelle un énoncé, même heuristique, qui engloberait ces deux théorèmes 
(dont le premier pour l’instant reste conjectural). Il manque apparement une idée nouvelle, comme 
il en manque aussi en Géométrie Algébrique pour parvenir à une démonstration du théorème de Riemann-Roch 
en dehors d’hypothèses projectives (cf. Exp. XIV, n$^\circ 2$).

Nous n’aurons à faire nul usage dans ce séminaire de la \emph{théorie locale} des intersections, exposée 
dans J. P. Serre, Algèbre Locale. Multiplicités (Lecture Notes in Mathematics $n^\circ$ 11, Springer), 
et nous contenterons simplement de signaler ici que ce cours de Serre a eu une influence évidente sur la 
genèse de la théorie en 1957. Nous n’utiliserons pas non plus la théorie de \emph{l’anneau de Chow} 
développée dans Séminaire Chevalley 1958, exposés 2 et 3 (École Normale Supérieure). Cette théorie est 
liée de façon essentielle à des hypothèses de non singularité, alors que le but de notre séminaire est 
au contraire de développer une théorie des intersections sur les schémas généraux (et même les topos 
localement annelés généraux) voir à ce sujet les commentaires dans Exp. XIV $n^\circ$ 4 et $n^\circ$ 8, 
donnant les relations entre notre théorie et celle de l’anneau de Chow, et posant la question d’une 
généralisation de cette dernière. Nous pouvons dire que le Séminaire présente une théorie des intersections 
cohérente et se suffisant à elle même, mais nullement exhaustive des différents points de vue utiles (voire 
indispensables) en théorie des intersections, et qu’il convient par suite de le compléter par le compléter 
par les exposés cites de Serre et de Chevalley.

Nous avons joint au Séminaire (en Appendice à Exp. 0) le rapport Grothendieck de 1957 "classes de Faisceaux 
et théorème de Riemann-Roch", qui avait servi de base au séminaire de Borel et Serre à Princeton la même année, 
ainsi qu’à leur article déjà cité. Ce rapport esquisse deux démonstrations de théorème de Riemann-Roch pour les 
variétés quasi-projectives non singulières, dont la première, valable pour le moment seulement en caractéristique 
nulle, mais donnant en revanche un résultat un peu plus précis dans le cas d’une immersion, ne figure pas encore 
dans la littérature (sans exclure le travail de Borel-Serre ni le présent séminaire). La lecture de ce rapport 
ne présuppose bien entendu celle d’aucun autre exposé du Séminaire, et peut même servir d’introduction à l’étude 
de ce dernier  au même titre que l’exposé 0, surtout pour un lecteur qui ne serait pas encore familier avec 
Borel-Serre. La démonstration à laquelle nous venons de faire allusion s’applique essentiellement telle quelle 
au cas d’un morphisme projectif d’espaces analytiques complexes, et s’appliquera sans doute également en 
caractéristique quelconque, une fois résolu le problème des opérations "puissances extérieures" dans la 
catégorie dérivée (cf. Exp. XIV, n$^\circ$ 1, 2, 3). C’est l’absence d’une étude de cette question qui constitue 
sans doute la lacune la plus sérieuse dans ce Séminaire, dans l’optique même où nous nous y sommes placés.

On remarquera l’absence, dans la table de matières du présent Séminaire, de deux des participants mentionnés 
sur la page de garde. L’un, J. P. Jouanolou, avait pris une part active à l’élaboration technique de la première 
partie de Séminaire, mais avait été empêché de prendre part aux exposés oraux ; on lui doit notamment 
l’assertion d’indépendance linéaire contenue dans l’important énoncé VI 1.1 donnant la structure de l’anneau 
$K^\bullet$ des classes de Modules localement libres sur un fibré projectif (qui n’était prouvé auparavant 
que lorsqu’on supposait que le schéma de base admet un Module inversible ample). L’autre, J. P. Serre, avait 
fait deux exposés oraux, logiquement indépendants du reste du Séminaire, et qu’il a par suite préféré publier 
sous forme d’article séparé (cf. J. P. Serre, Groupes de Grothendieck des schémas en groupes réductifs déployés 
à paraître dans Publications Mathématiques, $n^\circ$ 34).

Comme dans les autres parties du Séminaire de Géométrie Algébrique du Bois Marie, les sigles SGA 1 à SGA 6 
renvoient aux différentes parties dudit Séminaire, le chiffre romain suivant ce sigle indiquant le n$^\circ$ 
de l’exposé, et les chiffres arabes qui le suivent correspondant à la numérotation intérieure de l’exposé en 
question; pour les références intérieures au présente Séminaire, on omet le sigle SGA 6, en utilisant des références 
style I 4.9 (signifiant: exposé I, énoncé 4.9). Le sigle EGA réfère aux Éléments de Géométrie Algébrique de J. 
Dieudonné et A. Grothendieck.







%End
